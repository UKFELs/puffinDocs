\subsection{3D Magnetic Fields}

\label{und-fields}

\subsubsection{Main Undulator Models}

The undulator is modelled analytically, and the model must include the fast wiggle motion. Puffin may be modified in the future to allow a map of the undulator field to be input. For now, there are a few generic undulator models employed. The undulator magnetic fields are chosen with the {\bf zundType} string in the main input file.

Available options for use in Puffin are `helical', `planepole' - corresponding to a planar wiggler with flat pole faces with natural focusing only in one direction, `curved' - correspnding to a planar undulator with curved, or canted, pole faces providing natural focusing in both transverse directions equally, and the default Puffin undulator, chosen with a blank string (`'), where the transverse polarization is chosen with the {\bf sux, suy} inputs, controlling the relative magnitudes of the peak magnetic fields in $x$ and $y$, respectively. These may vary between $0$ and $1$, allowing a general elliptic field to be described.

All of the undulator fields have an associated `natural' focusing channel, which arises from from the off-axis variation in the magnetic fields. This motion arises naturally when numerically solving the equations, and is not super-imposed artificially upon the electron motion.

\subsubsection{Undulator Ends}

The undulators also include entry and exit tapers, and they may be switched on or off in the input file with the flag {\bf qUndEnds}. Setting this to true will model a smooth taper up and down of the undulator magnetic fields in the first and last 2 periods of the undulator, taking the form of a $\cos^2$. If they are switched off, the beam is artificially initialized with an `expected' initial condition in the transverse coordinates for that undulator. Including these ends will model a more realistic and natural entry and exit from the undulator, and will reduce CSE effects from the shape of the wiggler.



\subsubsection{Natural Undulator Focusing}

Each undulator type has an associated natural focusing wavenumber. In the helical case, the natural betatron wavenumber is
\begin{align}
\bar{k}_{\beta n x} = \bar{k}_{\beta n y} = \frac{a_w}{2 \sqrt{2} \rho \gamma_0},
\end{align}
with $\gamma_0$ being the average energy of the electron beam (and not necessarily = $\gamma_r$, which only sets the scaling of the system.)

In the planar case,
\begin{align}
\bar{k}_{\beta n y} = 0, \\
\bar{k}_{\beta n y} = \frac{a_w}{2 \sqrt{2} \rho \gamma_0}.
\end{align}

In the canted pole case,
\begin{align}
\bar{k}_{\beta n x,y} = \frac{a_w \bar{k}_{x,y}}{\sqrt{2 \eta} \gamma_0},
\end{align}
where $\bar{k}_{x,y}$ describe the hyperbolic variation in the transverse directions (see eqns (\ref{cp1} - \ref{cp3})), and must obey
\begin{align}
\bar{k}_x^2 + \bar{k}_y^2 = \frac{\eta}{4 \rho^2}
\end{align}
to be physically valid. They determine the focusing strength in the $\bar{x}$ and $\bar{y}$ dimensions. For the case of equal focusing, then,
\begin{align}
\bar{k}_{\beta n x} = \bar{k}_{\beta n y} = \frac{a_w }{ 4 \rho \gamma_0}.
\end{align}



\subsubsection{Strong Beam Focusing}

In addition to the natural focusing channel, a constant, `strong' focussing channel may be utilized, to focus the beam to a smaller transverse area. This is a magnetic field super-imposed upon the wiggler. It may be switched on or off with the flag {\bf qFocussing} in the main input file, and is specified through the use of the variables {\bf sKBetaXSF} and {\bf sKBetaYSF}. It is probably highly artificial - it may be thought of as physically similar to an ion channel. Nevertheless it allows one to obtain strong focusing without using a lattice. It is defined very simply as
\begin{align}
b_x = \sqrt{\eta} \frac{\bar{k}_{\beta y}^2}{\kappa}\bar{y}_j, \\
b_y = - \sqrt{\eta} \frac{\bar{k}_{\beta x}^2}{\kappa}\bar{x}_j
\end{align}

If either {\bf sKBetaXSF} or {\bf sKBetaYSF} are not specified, then no focusing channel will be added for that dimension, even if the {\bf qFocussing} flag is true.

Magnetic quads between modules can be specified in the lattice file. See section \ref{latt-file}.