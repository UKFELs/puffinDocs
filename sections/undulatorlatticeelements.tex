
\subsection{Undulator Line Elements}

Common lattice elements employed on the undulator line to focus or delay the beam are currently included as point transforms only. They are specified in the lattice file by their 2 letter identifier, followed by their required parameters.

\subsubsection{Quadrupole}

Identifier: {\bf QU}

For the quad, we must specify $\bar{F}_x$ and $\bar{F}_y$. These are scaled transverse components of the transform matrix for the quad. The quad is only a point transform, currently, and does not have a length, so the length must be included in a drift section. The transforms for the quad, in S.I. units, are defined as
\begin{align}
\frac{d x_j}{d z}  \bigg|_{new} = \frac{d x_j}{d z} \bigg|_{old} + \frac{1}{F_x} x_j \\
\frac{d y_j}{d z}  \bigg|_{new} = \frac{d y_j}{d z} \bigg|_{old} + \frac{1}{F_y} y_j
\end{align}
which in the scaled notation becomes:
\begin{align}
\frac{d \bar{p}_{xj}}{d \bar{z}} \bigg|_{new} = \frac{d \bar{p}_{xj}}{d \bar{z}}  \bigg|_{old} +\frac{\sqrt{\eta}}{2\rho\kappa} \frac{\bar{x}_j}{\bar{F}_x} \\
\frac{d \bar{p}_{yj}}{d \bar{z}}  \bigg|_{new} = \frac{d \bar{p}_{yj}}{d \bar{z}}  \bigg|_{old} +\frac{\sqrt{\eta}}{2\rho\kappa} \frac{\bar{y}_j}{\bar{F}_y}
\end{align}
where $\bar{F}_{x,y} = \dfrac{F_{x,y}}{l_g}$.

In $x$ (unscaled) focal length is taken as
\begin{align}
\frac{1}{F_x} = \frac{g L_Q}{B\rho},
\end{align}
where $g$ is the quad magnetic field gradient in $Tm^{-1}$, $L_Q$ is the length of the quad, and $B\rho = \beta E_0 / 0.2998$ is the so-called magnetic rigidity in $Tm$, with the beam energy $E_0$ in $GeV$ (corresponding to the scaling energy defined in the input file).

\subsubsection{Chicane}

Identifier: {\bf CH}

For the chicane, one must specify physical length, slippage length, and dispersion enhancement (scaled R56). The length is the physical length of the device in undulator periods, and is used to calculate the diffraction properly. The slippage length tells how many wavelengths to delay the beam by relative to the radiation field in resonant radiation wavelengths - be careful, this should not be $\leq 0$, or the beam will then be faster than light speed! Similarly, it is left up to the user to decide what is appropriate to delay the beam by with respect to the physical length of the device. The dispersive enhancement, $D$, is the scaled R56 of the device - using $\gamma_r$ in the main input as the mean $\gamma$ for the chicane, so that
\begin{align}
\bar{z}_{2} = \bar{z}_{20} - 2D \left( \frac{\gamma - \gamma_r}{\gamma_r}  \right),
\end{align}
and $D = k_r \rho  R_{56}$.

This approach of being able to vary the 3 parameters above is quite flexible, but at the expense of easy automatic checking of the physical relevance of the parameters. For example, one can disperse the beam 'in place', without shifting it w.r.t. the resonant wavelength, which is obviously unphysical. However, this may be useful for a quick setup of the beam for EEHG at the beginning of the undulator, for instance. If one is not sure of the physical length of the actual device, but knows the length between modules and the desired slippage (e.g. for mode locking the FEL), then one can specify the spatial drift between modules using a drift section, and then specify a chicane with zero physical length, but with the desired delay and R56. In this way the radiation diffraction will still be modelled correctly.

\subsubsection{Drift}

Identifier: {\bf DR}

For the drift, one must specify length in units of the number of undulator periods.

\subsubsection{Modulation}

Identifier: {\bf MO}

For the modulation section, one may add an energy modulation to the beam specified by $\bar{k}_M = \dfrac{2 \pi}{\bar{\lambda}_M}$ and $\dfrac{\Delta \gamma}{\gamma_r}$.
