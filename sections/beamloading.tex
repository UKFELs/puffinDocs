\subsection{Beam Initialization}
\label{beamfls}
There are 3 different ways of defining and initializing the electron beam in Puffin:

\subsubsection{Simple}

The beam is described in terms of a homogeneous Gaussian function in every dimension. Some simple correlations in energy can be achieved by specifying an oscillation in the beam energy as a function of $\bar{z}_2$, or as a simple linear energy chirp in $\bar{z}_2$. Twiss parameters may be used to tilt the beam for matching to a FODO lattice (Puffin does not calculate the matched Twiss parameters - this must be done separate to Puffin). The beam is generated in Puffin according to this description.

In general, the correct noise statistics are added to the beam with the method described in \cite{noise1}. This method, like other beam noise algorithms for FELs, requires a quiet beam to add the noise to. We include two methods of generating the beam - one with the correct noise in all dimensions, and one with the correct noise only in the temporal/$\bar{z}_2$ dimension. In both methods, the beam is initially quiet in the temporal dimension before adding the noise, with an equispaced layout of the particles in this dimension. The beam MUST be created to appropriately sample the wavelength of emission/amplification - so at least 10 equispaced particles per resonant wavelength. The methods are:-

% qSI - input in SI units - so x, y, z, px, py and gamma - system is then scaled to rho
% Rho and charge are decoupled - so the simulation will always be correct. For the scaled notation to make sense, the rho must be approximately correct for the given beam charge.
% qEqui = .false. by default.



{\bf Equispaced Grid in Every Dimension}

When using this option, the beam is intialized on an equispaced 6D grid. This requires very many particles to appropriately sample every dimension, and you may find it very easy to have more macroparticles than real electrons while loading the beam this way. However, we leave it up to the user to decide if this is appropriate - for some extreme dispersion situation with high charge it may be necessary to create the beam in this way.  This can be activated by setting {\bf qEquiXY $=$ .true.} in the beam input. {\bf qEquiXY} is actually an array of size nbeams, with a seperate value for each beam if multiple beams are desired. {\bf qEquiXY $=$ .false.} by default.

{\bf Equispaced grid ONLY in $\bar{z}_2$}

If qEquiXY is false, then the beam is intially generated as a 1D beam, with equispaced macroparticles. Each macroparticle is then split into many particles in the other 5 dimensions, with coordinates generated by a quasi-random sequence in each dimension. The SAME SEQUENCE is used for each particle in z2 - the creates a series of beamlets in the $\bar{z}_2$ dimension, giving a quiet start in $\bar{z}_2$. Because random sequences are used, the other 5-dimensions require orders of magnitude less particles to adequately fill the phase space than in the equispaced case. This is the default option, so if not specified, {\bf qEquiXY $=$ .false.}.

Both of the above methods may, of course, be replicated and modified by the user to suit a particular situation. The creation of the beam with the correct statistics, while still retaining the sampling necessary for the FEL interaction, is still an area of ongoing research, and is quite controversial, especially when including large velocity disperion in the beam dynamics. We ultimately leave it to the user to ensure the beam is appropriately initialized for their situation. The methods above can probably be replicated with just a few lines of \textit{e.g.} Python. The generated beam can then be input into Puffin.

To use this, set the option dtype = `simple' in the nblist namelist in the beam file. The parameters for the distribution are then set in the blist namelist in the same file.

\subsubsection{Dist input}

The input is composed of a description of temporal slices along the bunch, describing a Gaussian mean and standard deviation in every other dimension, along with Twiss $\alpha$ parameter, for each temporal slice. This information is used to generate the electron beam in Puffin. The slices must be spaced finely enough to sample the radiation field wavelength being amplified - so \textit{e.g} at least 10 per wavelength.

\subsubsection{Beam input}

This method allows one to read in the 6D scaled particle coordinates from an HDF5 file or a text file. The HDF5 file should be in the same format as the Puffin output. The text input is not recommended, especially for 3D runs, since the reading in is very inefficient, and the files are usually huge for 3D data. The beam is generated externally, and the user is then responsible for ensuring the sampling and the noise in the pulse are correct - it is not added by Puffin in this method.